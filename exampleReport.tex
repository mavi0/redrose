\documentclass[]{scrreprt}

\usepackage{Redrose}

\title{PhD Supervisor Meeting}
\author{Thomas Swarbrick\footnote{Supervised by Dr Haris Rotos \& Prof Nicholas Race.}\footnote{This research was funded by AI4ME, a BBC prosperity partnership.}}
\date{\today}


\bibliography{testing}


\begin{document}
\maketitle

\chapter{Ongoing  Reading}
\section{Covering Additional Content}
As part of the PhD Bi-weekly meeting I have continued to read around the subject area and learn more about the subject, I have particularly focused on the areas outlined below.
\subsection{365 preparation.}

In preparation for the 365 module next term I have undertook the following reading in the inter-rim period:
\begin{itemize}
  \item Re-watched the 365 lecture content from last year, make new sets of notes, convert notes into Anki Flashcards, introduced into spaced repetition learning regimen.
  \item Worked on the coursework implementation/Ryu tutorials from last year to be confident enough to TA/Assist in the delivery of labs for the 2024/2025 academic year.
  \item Additional reading from recommended reading list that may be listed in the module such as:
        \begin{itemize}
          \item Book referenced in the moodle page.
          \item Important papers that are directly related.
        \end{itemize}
\end{itemize}
\subsection{GPU/Encoding reading.}

I have began an implentation of my idea by begining to learn how to run OpenCL kernels on the graphics card on the fiona system (A40).
\begin{itemize}
  \item Implementation of the DCT2 II /DCT2 III in kl/c97 for further understanding.
\end{itemize}
\subsection{maths understanding.}
Working through additional mathematics to further my understanding of mathematics, making use of both the MIT Opencourseware Mathematics and Books on Information theory from the library.
\begin{itemize}
  \item Re-freshers on proofs, number theory \& graph theory.
  \item Reading of Information theory, along with encoding ideas by Thomas M. Cover et al.
\end{itemize}


\chapter{$S_{x}$ Breakout }
\section{Diagramed Work from Ed Suggestion}
\section{Approaches  to cross-correlation}

\chapter{Additional Learning}
\section{Beuwulf Cluster}
\subsection{Ansible}
\subsection{K8s}
\section{\LaTeX}
\subsection{Beamer}
\subsection{Report}

\chapter{Reading}
\section{Networking related textbooks/concepts to explore.}
\section{Unsorted papers to read.}
\end{document}
