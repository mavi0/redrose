\documentclass[aspectratio=169,nocontents]{beamer}

% Options are:
% - nocontents -> don't show the contents page.
% - bib -> show a bib on title.


\usetheme{Redrose}

\module{XX: Formalised Networks} % need to add this
\role{Researcher}
\email{f.surname}
% Need to acknowledge? It's not needed
\acknowledge{\textbf{v1.redrose} \LaTeX, coded by  \href{https://github.com/lancs-net}{tom@nsrg}}
\title{}
\author{Jane Doe}

\begin{document}
\maketitle
\section{And here we can put}
\begin{frame}[t]{Frame title goes here}
  \subsection{A clickable breakdown}
  \begin{exampleblock}{The Discrete Noiseless Channel}
    In the more general case with different length of symbols and constraints on the allowed sequences, we make the following definition:
    \subsection{of each individual thing in the pdf}
    \begin{quote}
      The capacity $C$ of a discrete channel is given by
    \end{quote}
    \[ C = \lim_{T \to \infty}  \frac{logN(T)}{T}\]

    \begin{quote}
    where N(T) is the number of allowed signals of duration $T$.
  \end{quote}
\end{exampleblock}
\subsection{Along with references to view}
  \begin{theorem}
    \[C = \lim_{T \to \infty}\frac{LogAX^{T}_{0}}{T} = logX_{0}\]
  \end{theorem}
\end{frame}
\end{document}
